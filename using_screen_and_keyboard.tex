%----------------------------------------------------------------------------------------
%	PART - Using the Screen and Keyboard
%----------------------------------------------------------------------------------------

\makeatletter\@openrightfalse
\part{Using the Screen and Keyboard}

\chaptertypein{
	\keybackgroundcolor{gray}
	\keytextcolor{black}
	10 PRINT "\shiftkey\clrhomekey"\\
	20 SLEEP 10\\
	30 PRINT "your name"\\
	40 SLEEP 10\\
	50 GOTO 10
}

\newpage
\partheading{Using the Screen and Keyboard}

This chapter assumes that you've read and understand Chapter 1: {\emph Getting
to Know Your Commander X16}.  If you not, go back and read it.\\

Now that you have written your first program in BASIC on the Commander X16,
it's time to look at the simplest thing you can make the X16 do: displaying
characters to the screen.  As you've already seen, the {\ttfamily PRINT}
statement can be followed by a series of character surrounded by quotation
marks, causing the series of charcters to be displayed to the screen.  This
series of characters is called a {\em string}.  Let's write a program to print
a string to the screen.\\

It's nice to start a new program on an empty screen, so lets clear the screen
and place the cursor at the top by holding down the \shiftkey key and pressing
the \clrhomekey key.  To tell the Commander X16 that you are writing a new
program, type \keybackgroundcolor{white}\key{n}\key{e}\key{w} and then press
the \returnkey key.  Now you can begin to type the following program, paying
attention to line number and punctuation marks:\\

\key{1}\key{0} \spacebar \key{p}\key{r}\key{i}\key{n}\key{t} \spacebar
\key{"}\key{h}\key{e}\key{l}\key{l}\key{o}\key{,} \spacebar
\key{w}\key{o}\key{r}\key{l}\key{d}\key{!}\key{"} \returnkey

\note {

	The \key{,} key is just the \doublekey{<\\,} key pressed {\em without}
	holding the \shiftkey key, and the \key{!} key is just the \doublekey{!\\1}
	key pressed {\em while} holding the \shiftkey key.

}

Now that your program is typed into the X16, run it by typing
\key{r}\key{u}\key{n} and pressing \returnkey.

You should see the following printed to the next line:

\codeblock {
	HELLO, WORLD!\\
}

After your program finishes printing out the message, it should print a blank
line, a line with the {\ttfamily READY} prompt, and finally place your blinking
cursor on the following line.  Your screen should look something like this:\\

\begin{center}
	\screenbox{2.75in}{2in}{

		NEW\\
		
		READY.\\
		10 PRINT "HELLO, WORLD"\\
		RUN\\
		HELLO, WORLD!\\

		READY.\\
		\cursor

	}
\end{center}

When you typed {\ttfamily NEW} and hit \returnkey, the Commander X16 also left
a blank line followed by the {\ttfamily READY} prompt, just as it did when your
own program ran.  You can think of BASIC commands like {\ttfamily NEW} as being
built-in programs, and they act much in the same way as the programs you will
write.\\

%----------------------------------------------------------------------------------------
%	CHAPTER - The PRINT Statement
%----------------------------------------------------------------------------------------

\chapter*{The PRINT Statement}
\addcontentsline{toc}{chapter}{\protect\numberline{}The PRINT Statement}

Now that we've used the {\ttfamily PRINT} statement to display a string of
letters to the screen, let's look at how it can display numbers.  You can
re-use your existing program by overwriting the statement at line 10.  Do do
this, simply type {\ttfamily 10} followed by the BASIC statement you want to
replace line 10 with.  In this case, let's replace it with a {\ttfamily PRINT}
statement that prints numbers instead of letters:\\

\codeblock {

	10 PRINT "42"\\

}

Make sure you press \returnkey so that the Commander X16 accepts the new line,
and replaces the old one in your program.  Now if you type {\ttfamily RUN} and
hit \returnkey, you should see:

\codeblock{
	42\\
}

This example isn't much different than the last example, because we simply
printed a different string.  The number {\ttfamily 42} is given to the
{\ttfamily PRINT} statement as a string, just as {\ttfamily HELLO, WORLD!} was
given as a string.  To the {\ttfamily PRINT} statement, it doesn't matter which
characters are put inside the quotations marks; it always views them as just a
series of characters.  However, the {\ttfamily PRINT} statement is able to
display a number to the screen while treating it as a number.  For example,
replace line 10 with:\\

\codeblock {

	10 PRINT 42\\

}

Notice that there are no quotation marks.  When type {\ttfamily RUN}, you will
notice a slight difference:\\

\codeblock{
	\hspace*{0.6em}42\\
}

Do you see it?  There is an empty space before the number.  This is because the
{\ttfamily PRINT} statement leaves a space for a sign indicator (positive or
negative) when it prints a number.  For positive numbers nothing is displayed,
but if given a negative number, {\ttfamily PRINT} will fill that space with a
{\ttfamily -}.  Try replacing {\ttfamily 42} with {\ttfamily -42} and running
the program again.\\

Now that you have the {\ttfamily PRINT} statement treating the value as a
number, you can try passing it a mathematical expression that the X16 will {\em
compute} into a number:\\

\codeblock {

	10 PRINT 3+2\\

}

You should see:\\

\codeblock {
	\hspace{0.6em}5\\
}

Now you can compare this result with what happens if we pass {\ttfamily 3+2} to
{\ttfamily PRINT} as a string:\\

\codeblock {

	10 PRINT "3+2"\\

}

This time you should see:\\

\codeblock {
	3+2\\
}

When you pass a value to {\ttfamily PRINT} inside of quotation marks, it treats
it as a series of characters, Specifically "3", "+", and "2".  {\ttfamily
PRINT} is also smart enough to accept a mixture of strings and numeric
values:\\

\codeblock {

	10 PRINT "THERE ARE" 5+3 "BITS IN A BYTE"\\

}

%----------------------------------------------------------------------------------------
%	CHAPTER - Graphic Characters
%----------------------------------------------------------------------------------------

\chapter*{Graphic Chracters}
\addcontentsline{toc}{chapter}{\protect\numberline{}Graphic Chracters}

Sit down in front of the Commander X16 keyboard and look at the keys.  If
you're familiar with regular computer or typewriter keyboard, you'll notice
that the X16 Keyboard's keys are different.  Rather than just having letters,
numbers, and punctuation on them, you'll see that most keys also have one or
more "graphic characters" printed on them.  While the regular characters can be
typed by simply pressing the keys, the graphics characters can be typed by
holding down either \shiftkey or \altkey and pressing one of the keys with
graphics characters.  For example, try holding down the \shiftkey key and
pressing the \keybackgroundcolor{white}\key{s} key.  Instead of an 'S', a heart
should appear on the screen.  Now press and hold down the \altkey key and press
\key{s}.  This time a right-angled graphic character should appear.  Holding
the \shiftkey key causes the graphic character on the right to be typed, and
holding the \altkey key causes the graphic character on the left to be typed.\\

Graphic characters can be used to create larger shapes and images on the
screen.  For example, the graphic characters that can be typed by holding
\altkey and typeing \key{a}, \key{s}, \key{z}, and \key{x} can be used to
create a square that is larger than a single character.  If you include the
graphic characters that can be typed by holding \shiftkey and typing \key{c}
and \key{b}, you can draw a rectangle as large as the entire screen!  If you
want your rectangle to have rounded corners, you can use the graphic characters
on the \key{u}, \key{i}, \key{j}, and \key{k}.\\

Spend some time experimenting with the other graphic characters to see what
kinds of shapes and designs you can draw.  You can use the arrow keys to
position the cursor wherever you need to type.\\

% To start, sit down in front of the Commander X16 Keyboard and type the
% following:\\
% 
% Hold down the \shiftkey key and press the \clrhomekey key.  Then type
% \keybackgroundcolor{white}\key{N}\key{E}\key{W} and press the \returnkey key.\\
% 
% This will clear any program currently programmed so that the X16 is ready to
% accept a new one.

%----------------------------------------------------------------------------------------
%	CHAPTER - Colors
%----------------------------------------------------------------------------------------

\chapter*{Colors}
\addcontentsline{toc}{chapter}{\protect\numberline{}Colors}

Sapiente voluptas aut accusamus. Distinctio et reiciendis ut qui aliquam fuga.
Possimus cumque dolores non. Tempora facilis ratione ut cupiditate ducimus
tenetur laborum vel. Eligendi sapiente maxime temporibus temporibus qui
suscipit. Tempora omnis exercitationem rem perspiciatis sunt. Dolorum quisquam
est excepturi est a magni consequatur. Animi non ea provident. Aut nihil non
ducimus dolorem voluptatibus eius quos. Distinctio placeat cupiditate
consequatur totam reprehenderit nihil est. Voluptatibus voluptatem quam veniam
at et. Nobis quos voluptatibus labore autem. Mollitia sed iusto hic eius et.
Nobis eaque qui id aliquam odit id. Consequatur quis et eos sint dolor eum
omnis quo. Quod voluptatem consequatur odio. Dicta vero numquam minus. Aut
maiores est molestiae eligendi occaecati rem suscipit eos. Et maxime aliquam
voluptatem. Voluptatem sed neque est. Quia ut omnis exercitationem aperiam
tenetur perferendis qui. Qui magni alias consequatur voluptatum rerum a.
Exercitationem et magnam nulla odio blanditiis voluptas cum quibusdam. Quo
tenetur animi autem quia alias vitae dolorem. Numquam error vero voluptatum.

%----------------------------------------------------------------------------------------
%CHAPTER - Keyboard
%----------------------------------------------------------------------------------------

\chapter*{The X16 Keyboard}
\addcontentsline{toc}{chapter}{\protect\numberline{}The X16 Keyboard}

Molestiae ad dicta praesentium et. Placeat magnam nihil est animi vel eos. Sunt
consectetur nobis minima ut reiciendis hic non sed. Officiis sint voluptas non
quo eos architecto. Nulla et est laboriosam voluptatem. Iure sed et ducimus
nostrum est eveniet. Natus aut praesentium fugit. In quae tempora sunt autem
illum perspiciatis. Amet laborum numquam aut occaecati. Quia ad ab voluptas qui
autem. Qui voluptatum quibusdam est aliquam in. Quae ipsum aperiam aut saepe
molestiae natus sit. Totam autem veritatis deserunt. Hic ut excepturi porro. Et
ut vero voluptas iusto earum velit rerum. Assumenda enim voluptatum praesentium
quam. Rerum optio iste odit. Id quia ratione quasi. Doloremque et omnis autem
dolor. Vel minima numquam enim asperiores quae magni soluta a. Corrupti sint
sit sunt cum sunt asperiores animi rerum. Consectetur sunt itaque ducimus
soluta sed quod qui. Blanditiis alias rem ea. Doloremque nobis voluptas eius
occaecati mollitia temporibus enim ut. Quia consequuntur molestias quae modi
consequatur eveniet consequuntur.

%----------------------------------------------------------------------------------------
%CHAPTER - Screen Modes
%----------------------------------------------------------------------------------------

\chapter*{Screen Modes}
\addcontentsline{toc}{chapter}{\protect\numberline{}Screen Modes}

Minima ut quasi aliquid sapiente quo. Id veritatis ipsa vitae molestias velit
modi natus et. Quia incidunt totam laboriosam nostrum sed nihil. Assumenda
reiciendis molestiae quidem enim quis. Eius excepturi neque dolorem quia.
Nesciunt et consequuntur expedita enim soluta in recusandae. Reprehenderit
rerum ut facilis aut eius. Velit eveniet esse dolorem dolores tempore ratione
tempora. Iste sequi architecto dolores repellendus rem quia sequi. Voluptas
error maxime ipsam est sint saepe. Nulla similique eos dolorem esse nobis autem
nam a. Aut fugit quae reprehenderit qui. Minima dolorum nihil sapiente dolorem
porro minima id. Perspiciatis natus numquam voluptatum. Qui sint nemo
praesentium exercitationem voluptatem esse. Et perferendis praesentium
voluptatibus. Occaecati facere eligendi eos eum exercitationem. Nobis aperiam
inventore laudantium eius consequatur cupiditate. Tempora dolore culpa magni ut
eum sint voluptatibus. Quasi repudiandae necessitatibus repellendus cumque quia
dolorum. Illo et ut qui. Ut alias et quod repellat sit nobis. Officiis
doloremque quaerat vitae iste. Et nostrum pariatur dolorum iusto nulla quae ab.
Et voluptatem itaque quae perspiciatis quia.

%----------------------------------------------------------------------------------------
%	CHAPTER - Editing Text
%----------------------------------------------------------------------------------------

\chapter*{Editing Text}
\addcontentsline{toc}{chapter}{\protect\numberline{}Editing Text}

Ea repudiandae laboriosam omnis consequatur omnis quam nesciunt est. Hic
voluptate explicabo sint pariatur mollitia. Omnis asperiores in praesentium
dolor quibusdam. Odit vitae possimus ut recusandae quia sapiente ipsum non.
Optio error velit eligendi. Facere itaque tenetur fugiat. Aspernatur magnam
tenetur nulla aspernatur architecto. Repellat repudiandae autem sunt qui.
Libero vel dolorem sed mollitia. Voluptatem eligendi minima voluptatum facere
aut magnam laborum vel. Quis rem officia aliquam nisi quisquam dolor quia.
Nobis magni non eos explicabo. Qui nisi in est voluptatem enim a repellendus.
Id quia incidunt enim sint impedit recusandae. Ut sit ut neque. Et sit delectus
id excepturi. Vero maiores libero minus. Ad accusantium sed ut vitae quia
earum. Iusto fugit repudiandae aut. Illum atque et dolores. Suscipit qui culpa
dolor. Voluptatibus consequuntur culpa ab vitae. Totam libero harum quia. Ipsa
aut minima labore eaque eos ipsam. Nulla quae eius tempore.

\@openrighttrue\makeatother
