\chapter*{YM2151 Registers}
\addcontentsline{toc}{chapter}{\protect\numberline{}YM2151 Registers}

The YM register address space can be thought of as being divided into 3 ranges:\\

\begin{tabular}{|c|c|p{0.5\linewidth}|}
	\hline
	Range & Type & Description \\ \hline
	\$00 .. \$1F & Global Values & Affect individual global parameters such as LFO frequency, noise enable, etc. \\ \hline
	\$20 .. \$3F & Channel CFG & Parameters in groups of 8, one per channel. These affect the whole channel. \\ \hline
	\$40 .. \$FF & Operator CFG & Parameters in groups of 32 - these map to individual operators of each voice. \\ \hline
\end{tabular}\\

\subsection{Global Registers}

\begin{longtable}{|m{0.10\linewidth}|m{0.2\linewidth}|c|c|c|c|c|c|c|c|c|}
	\hline

	\multirow{2}{*}{Addr}&\multirow{2}{*}{Register}&\multicolumn{8}{c|}{Bits}\\\cline{3-10}
	& & 7 & 6 & 5 & 4 & 3 & 2 & 1 & 0 \\ \hline

	\multirow{2}{*}{\$01} & \multirow{2}{*}{Test} & ! & ! & ! & ! & ! & ! & LR & ! \\\cline{3-10}

	& & \multicolumn{8}{p{0.55\linewidth}|} {Bit 1 is the LFO reset bit.
	Setting it disables the LFO and holds the oscillator at 0. Clearing it
	enables the LFO.  All other bits control various test functions and should
	not be written into.}\\ \hline

	\multirow{2}{*}{\$08} & \multirow{2}{*}{Key Control} & . & C2 & M2 & C1 & M1 & \multicolumn{3}{c|}{CHA} \\\cline{3-10}

	& & \multicolumn{8}{p{0.55\linewidth}|} {Starts and Releases notes on the 8
	channels.  Setting/Clearing bits for M1,C1,M2,C2 controls the key state for
	those operators on channel CHA.  NOTE: The operator order is different than
	the order they appear in the Operator configuration registers!}\\ \hline

	\multirow{2}{*}{\$0F} & \multirow{2}{*}{Noise Control} & NE & . & . &  \multicolumn{5}{c|}{NFRQ} \\\cline{3-10}

	& & \multicolumn{8}{p{0.55\linewidth}|} {NE = Noise Enable; NFRQ = Noise
	Frequency;  When eabled, C2 of channel 7 will use a noise waveform instead
	of a sine waveform.  }\\ \hline

	\multirow{2}{*}{\$10} & \multirow{2}{*}{Timer A High} & \multicolumn{8}{c|}{CLKA1} \\\cline{3-10}

	& & \multicolumn{8}{p{0.55\linewidth}|} {Top 8 bits of Timer A period setting}\\ \hline

	\multirow{2}{*}{\$11} & \multirow{2}{*}{Timer A Low} & . & . & . & . & . & . & \multicolumn{2}{c|}{CLKA2} \\\cline{3-10}

	& & \multicolumn{8}{p{0.55\linewidth}|} {Bottom 2 bits of Timer A period setting}\\ \hline

	\multirow{2}{*}{\$12} & \multirow{2}{*}{Timer B} & \multicolumn{8}{c|}{CLKB} \\\cline{3-10}

	& & \multicolumn{8}{p{0.55\linewidth}|} {Timer B period setting}\\ \hline

	\multirow{2}{*}{\$14} & \multirow{2}{*}{IRQ Control} & CSM & . &
	\multicolumn{2}{c|}{CLK ACK} &
	\multicolumn{2}{c|}{IRQ EN} &
	\multicolumn{2}{c|}{CLK ST} \\\cline{3-10}

	& & \multicolumn{8}{p{0.55\linewidth}|} {CSM: When a timer expires, trigger
	note key-on for all channels.  For the other 3 fields, lower bit = Timer A,
	upper bit = Timer B.  CLK ACK: clears the timer's bit in the YM\_status
	byte and acknowledges the IRQ.}\\ \hline

	\multirow{2}{*}{\$18} & \multirow{2}{*}{LFO Freq} & \multicolumn{8}{c|}{LFRQ} \\\cline{3-10}

	& & \multicolumn{8}{p{0.55\linewidth}|} {Sets LFO frequency.  \$00 =
	~0.008Hz \$FF = ~32.6Hz}\\ \hline

	\multirow{3}{*}{\$19} & \multirow{3}{*}{LFO Amplitude} & 0 & \multicolumn{7}{c|}{AMD} \\ \cline{3-10}

	& & 1 & \multicolumn{7}{c|}{PMD} \\ \cline{3-10}

	& & \multicolumn{8}{p{0.55\linewidth}|} {AMD = Amplitude Modulation Depth;
	PMD = Phase Modulation (vibrato) Depth;  Bit 7 determines which parameter
	is being set when writing into this register.}\\ \hline

	\multirow{2}{*}{\$1B} & \multirow{2}{*}{LFO Waveform} &
	\multicolumn{2}{c|}{CT} & . & . & . & . &
	\multicolumn{2}{c|}{W} \\\cline{3-10}

	& & \multicolumn{8}{p{0.55\linewidth}|} {CT: sets output pins CT1 and CT1
	high or low. (not connected to anything in X16); W: LFO Waveform: 0-4 =
	Saw, Square, Triange, Noise; For sawtooth: PM->////
	AM->\textbackslash\textbackslash\textbackslash\textbackslash}\\ \hline

\end{longtable}

{\bfseries LR} (LFO Reset)

Register \$01, bit 1\\

Setting this bit will disable the LFO and hold it at level 0. Clearing this bit
allows the LFO to operate as normal. (See LFRQ for further info)\\

{\bfseries KON} (KeyON)

Register \$08\\

\begin{itemize}
	\item Bits 0-2: Channel\_Number

	\item Bits 3-6: Operator M1, C1, M2, C2 control bits:
		\begin{itemize}
			\item 0: Releases note on operator
			\item 0->1: Triggers note attack on operator
			\item 1->1: No effect
		\end{itemize}
\end{itemize}

\vspace{16pt}

Use this register to start/stop notes. Typically, all 4 operators are
triggered/released together at once. Writing a value of \$78+channel\_number
will start a note on all 4 OPs, and writing a value of \$00+channel\_number
will stop a note on all 4 OPs.\\

{\bfseries NE} (Noise Enable)

Register \$0F, Bit 7\\

When set, the C2 operator of channel 7 will use a noise waveform instead of a
sine.\\

{\bfseries NFRQ} (Noise Frequency)

Register \$0F, Bits 0-4\\

Sets the noise frequency, \$00 is the lowest and \$1F is the highest. NE bit
must be set in order for this to have any effect. Only affects operator C2 on
channel 7.\\

{\bfseries CLKA1} (Clock A, high order bits)

Register \$10, Bits 0-7\\

This is the high-order value for Clock A (a 10-bit value).\\

{\bfseries CLKA2} (Clock A, low order bits)

Register \$11, Bits 0-1\\

Sets the 2 low-order bits for Clock A (a 10-bit value).  Timer A's period is
Computed as:\\

\codeblock{
	(64*(1024-ClkA)) / PhiM ms.  (PhiM = 3579.545Khz)\\
}

{\bfseries CLKB} (Clock B)

Register \$12, Bits 0-7\\

Sets the Clock B period.  The period for Timer B is computed as:\\

\codeblock{
	(1024*(256-CLKB)) / PhiM ms. (PhiM = 3579.545Khz)\\
}

{\bfseries CSM}

Register \$14, Bit 7\\

When set, the YM2151 will generate a KeyON attack on all 8 channels whenever
Timer A overflows.\\

{\bfseries Clock ACK}

Register \$14, Bits 4-5\\

Clear (acknowledge) IRQ status generated by Timer A and Timer B
(respectively).\\

{\bfseries IRQ EN}

Register \$14, Bits 2-3\\

When set, enables IRQ generation when Timer A or Timer B (respectively)
overflow.  The IRQ status of the two timers is checked by reading from the
YM2151\_STATUS byte.  Bit 0 = Timer A IRQ status, and Bit 1 = Timer B IRQ
status. Note that these status bits are only active if the timer has overflowed
AND has its IRQ\_EN bit set.\\

{\bfseries Clock Start}

Register \$14, Bits 0-1\\

When set, these bits clear the Timer A and Timer B (respectively) counters and
starts it running.\\

{\bfseries LFRQ} (LFO Frequency)

Register \$18, Bits 0-7\\

Sets the LFO frequency:

\begin{itemize}
	\item \$00 = ~0.008Hz
	\item \$FF = ~32.6Hz
\end{itemize}

\vspace{16pt}

Note that even setting the value zero here results in a positive LFO frequency.
Any channels sensitive to the LFO will still be affected by the LFO unless the
{\ttfamily LR} bit is set in register \$01 to completely disable it.\\

{\bfseries AMD} (Amplitude Modulation Depth)

Register \$19 Bits 0-6, Bit 7 clear\\

Sets the peak strength of the LFO's Amplitude Modulation effect. Note that bit
7 of the value written into \$19 must be clear in order to set the AMD. If bit 7
is set, the write will be interpreted as PMD.\\

{\bfseries PMD} (Phase Modulation Depth)

Register \$19 Bits 0-6, Bit 7 set\\

Sets the peak strength of the LFO's Phase Modulation effect. Note that bit 7 of
the value written into \$19 must be set in order to set the PMD. If bit 7 is
clear, the value is interpreted as AMD.\\

{\bfseries CT} (Control pins)

Register \$1B, Bits 6-7\\

These bits set the electrical state of the two CT pins to on/off. These pins
are not connected to anything in the X16 and have no effect.\\

{\bfseries W} (LFO Waveform)

Register \$1B, Bits 0-1\\

Sets the LFO waveform:

\begin{itemize}
	\item 0: Sawtooth
	\item 1: Square (50\% duty cycle)
	\item 2: Triangle, 3: Noise
\end{itemize}

\vspace{16pt}


\clearpage

\subsection{Channel CFG Registers}

\begin{tabular}{|m{0.25\linewidth}|p{0.05\linewidth}|p{0.05\linewidth}|p{0.05\linewidth}|p{0.05\linewidth}|p{0.05\linewidth}|p{0.05\linewidth}|p{0.05\linewidth}|p{0.05\linewidth}|p{0.05\linewidth}|}
	\hline

	\multirowcell{2}{Register\\Range}&\multicolumn{8}{c|}{Register Bits} \\\cline{2-9}
	& 7 & 6 & 5 & 4 & 3 & 2 & 1 & 0 \\ \hline

	\$20 + channel & \multicolumn{2}{c|}{RL} & \multicolumn{3}{c|}{FB} & \multicolumn{3}{c|}{CON} \\\cline{1-9}

	\$28 + channel & . & \multicolumn{7}{c|}{KC} \\\cline{1-9}
	\$30 + channel & \multicolumn{6}{c|}{KF} & . & . \\\cline{1-9}
	\$38 + channel & . & \multicolumn{3}{c|}{PMS} & . & . & \multicolumn{2}{c|}{AMS} \\\cline{1-9}

	\multicolumn{9}{|c|}{Description} \\ \hline
	
	RL & \multicolumn{8}{c|}{Right/Left Output Enable}\\\hline
	FB &  \multicolumn{8}{c|}{M1 Feedback Level}\\\hline
	CON &  \multicolumn{8}{c|}{Operator Connection Algorithm}\\\hline
	KC & \multicolumn{8}{c|}{Key Code}\\\hline
	KF & \multicolumn{8}{c|}{Key Fraction}\\\hline
	PMS & \multicolumn{8}{c|}{Phase Modulation Sensitivity}\\\hline
	AMS & \multicolumn{8}{c|}{Amplitude Modulation Sensitivity}\\\hline

\end{tabular}

\vspace{16pt}

{\bfseries RL} (Right/Left output enable)

Register \$20 (+ channel), Bits 6-7\\

Setting/Clearing these bits enables/disables audio output for the selected
channel. (bit6=left, bit7=right)\\

{\bfseries FB} (M1 Self-Feedback)

Register \$20 (+ channel), bits 3-5\\

Sets the amount of self feedback on operator M1 for the selected channel.
0=none, 7=max\\

{\bfseries CON} (Connection Algorithm)

Register \$20 (+ channel), bits 0-2\\

Sets the selected channel to connect the 4 operators in one of 8 arrangements.\\

  % TODO [insert picture here]

{\bfseries KC} (Key Code - Note selection)

Register \$28 + channel, bits 0-6\\

Sets the octave and semitone for the selected channel.  Bits 4-6 specify the
octave (0-7) and bits 0-3 specify the semitone:

\begin{tabular}{|c|c|c|c|c|c|c|c|c|c|c|c|}
	\hline

	0&1&2&4&5&6&8&9&A&C&D&E\\ \hline

	{\ttfamily C♯}&{\ttfamily D}&{\ttfamily D♯}&{\ttfamily E}&{\ttfamily F}&{\ttfamily F♯}&{\ttfamily G}&{\ttfamily G♯}&{\ttfamily A}&{\ttfamily A♯}&{\ttfamily B}&{\ttfamily C}\\

	\hline
\end{tabular}

\vspace{16pt}

Note that natural C is at the TOP of the selected octave, and that each 4th
value is skipped. Thus if concert A (A-4, 440hz) is KC=\$4A, then middle C is
KC=\$3E\\

{\bfseries KF} (Key Fraction)

Register \$30 + channel, Bits 2-7\\

Raises the pitch by 1/64th of a semitone * the KF value.\\

{\bfseries PMS} (Phase Modulation Sensitivity)

Register \$38 + channel, Bits 4-6\\

Sets the Phase Modulation (vibrato) sensitivity of the selected channel. The
resulting vibrato depth is determined by the combination of the global PMD
setting (see above) modified by each channel's PMS.\\

Sensitivity values: (+/- cents)

\begin{tabular}{|c|c|c|c|c|c|c|c|}
	\hline

	0&1&2&3&4&5&6&7\\ \hline

	0&5&10&20&50&100&400&700\\

	\hline
\end{tabular}

\vspace{16pt}


{\bfseries AMS} (Amplitude Modulation Sensitivity)

Register \$38 + channel, Bits 0-1\\

Sets the Amplitude Modulation sensitivity of the selected channel. Note that
each operator may individually enable or disable this effect on its output by
setting/clearing the AMS-Ena bit (see below). Operators acting as outputs will
exhibit a tremolo effect (varying volume) and operators acting as modulators
will vary their effectiveness on the timbre when enabled for amplitude
modulation.\\

Sensitivity values: (dB)

\begin{tabular}{|c|c|c|c|}
	\hline

	0&1&2&3\\ \hline

	0&23.90625&47.8125&95.625\\

	\hline
\end{tabular}

\vspace{16pt}


\clearpage

\subsection{Operator CFG Registers}

\begin{longtable}{|m{0.10\linewidth}|m{0.25\linewidth}|p{0.033\linewidth}|p{0.033\linewidth}|p{0.033\linewidth}|p{0.033\linewidth}|p{0.033\linewidth}|p{0.033\linewidth}|p{0.033\linewidth}|p{0.033\linewidth}|p{0.033\linewidth}|}
	\hline

	\multirowcell{2}{Register\\Range} & \multirow{2}{*}{Operator} & \multicolumn{8}{c|}{Register Bits} \\\cline{3-10}
	& & 7 & 6 & 5 & 4 & 3 & 2 & 1 & 0 \\ \hline

	\multirow{4}{*}{\$40} & M1: \$40+channel &
	\multirow{4}{*}{.} &
	\multicolumn{3}{c|}{\multirow{4}{*}{DT1}} &
	\multicolumn{4}{c|}{\multirow{4}{*}{MUL}} \\
	& M2: \$48+channel & & \multicolumn{3}{c|}{} & \multicolumn{4}{c|}{} \\
	& C1: \$50+channel & & \multicolumn{3}{c|}{} & \multicolumn{4}{c|}{} \\
	& C2: \$58+channel & & \multicolumn{3}{c|}{} & \multicolumn{4}{c|}{} \\\hline

	\multirow{4}{*}{\$60} & M1: \$60+channel &
	\multirow{4}{*}{.} &
	\multicolumn{7}{c|}{\multirow{4}{*}{TL}} \\
	& M2: \$68+channel & & \multicolumn{7}{c|}{} \\
	& C1: \$70+channel & & \multicolumn{7}{c|}{} \\
	& C2: \$78+channel & & \multicolumn{7}{c|}{} \\\hline

	\multirow{4}{*}{\$80} & M1: \$80+channel &
	\multicolumn{2}{c|}{\multirow{4}{*}{KS}} &
	\multirow{4}{*}{.} &
	\multicolumn{5}{c|}{\multirow{4}{*}{AR}} \\
	& M2: \$88+channel & \multicolumn{2}{c|}{} & & \multicolumn{5}{c|}{}\\
	& C1: \$90+channel & \multicolumn{2}{c|}{} & & \multicolumn{5}{c|}{}\\
	& C2: \$98+channel & \multicolumn{2}{c|}{} & & \multicolumn{5}{c|}{}\\\hline

	\multirow{4}{*}{\$A0} & M1: \$A0+channel &
	\multirow{4}{*}{AM} &
	\multirow{4}{*}{.} &
	\multirow{4}{*}{.} &
	\multicolumn{5}{c|}{\multirow{4}{*}{D1R}} \\
	& M2: \$A8+channel & & & & \multicolumn{5}{c|}{}\\
	& C1: \$B0+channel & & & & \multicolumn{5}{c|}{}\\
	& C2: \$B8+channel & & & & \multicolumn{5}{c|}{}\\\hline

	\multirow{4}{*}{\$C0} & M1: \$C0+channel &
	\multicolumn{2}{c|}{\multirow{4}{*}{DT2}} &
	\multirow{4}{*}{.} &
	\multicolumn{5}{c|}{\multirow{4}{*}{D2R}} \\
	& M2: \$C8+channel & \multicolumn{2}{c|}{} & & \multicolumn{5}{c|}{}\\
	& C1: \$D0+channel & \multicolumn{2}{c|}{} & & \multicolumn{5}{c|}{}\\
	& C2: \$D8+channel & \multicolumn{2}{c|}{} & & \multicolumn{5}{c|}{}\\\hline

	\multirow{4}{*}{\$E0} & M1: \$E0+channel &
	\multicolumn{4}{c|}{\multirow{4}{*}{D1L}} &
	\multicolumn{4}{c|}{\multirow{4}{*}{RR}} \\
	& M2: \$E8+channel & \multicolumn{4}{c|}{} & \multicolumn{4}{c|}{} \\
	& C1: \$F0+channel & \multicolumn{4}{c|}{} & \multicolumn{4}{c|}{} \\
	& C2: \$F8+channel & \multicolumn{4}{c|}{} & \multicolumn{4}{c|}{} \\\hline

	\multicolumn{10}{|c|}{Description} \\ \hline
	
	DT1 & \multicolumn{9}{c|}{Detune Amount (fine)}\\\hline
	MUL & \multicolumn{9}{c|}{Frequency Multiplier}\\\hline
	TL & \multicolumn{9}{c|}{Total Level (volume attenuation) (0=max, \$7F=min)}\\\hline
	KS & \multicolumn{9}{c|}{Key Scaling (ADSR rate scaling)}\\\hline
	AR & \multicolumn{9}{c|}{Attack Rate}\\\hline
	AM & \multicolumn{9}{c|}{Amplitude Modulation Enable}\\\hline
	D1R & \multicolumn{9}{c|}{Decay Rate 1 (From peak down to sustain level)}\\\hline
	DT2 & \multicolumn{9}{c|}{Detune Amount (coarse)}\\\hline
	DR2 & \multicolumn{9}{c|}{Decay Rate 2 (During sustain phase)}\\\hline
	D1L & \multicolumn{9}{c|}{Decay Level 1 (Sustain level)}\\\hline
	RR & \multicolumn{9}{c|}{Release Rate}\\\hline

\end{longtable}

Operators are arranged as follows:

\begin{tabular}{|c|c|c|c|c|}
	\hline
	name&M1&M2&C1&C2 \\ \hline
	index&0&1&2&3 \\ \hline
\end{tabular}

\vspace{16pt}

These are the names used throughout this document for consistency, but they may
function as either modulators or carriers, depending on which {\ttfamily CON}
ALG is used.\\

The Operator Control parameters are mapped to channels/operators as follows:
Register + 8\*op + channel. You may also choose to think of these register
addresses as using bits 0-2 = channel, bits 3-4 = operator, and bits 5-7 =
parameter. This reference will refer to them using the address range, e.g.
\$60-\$7F = TL. To set TL for channel 2, operator 1, the register address would
be \$6A (\$60 + 1\*8 + 2).\\


{\bfseries DT1} (Detune 1 - fine detune)

Registers \$40-\$5F, Bits 4-6\\

Detunes the operator from the channel's main pitch. Values 0 and 4=no detuning.
Values 1-3=detune up, 5-7 = detune down.  The amount of detuning varies with
pitch. It decreases as the channel's pitch increases.\\

{\bfseries MUL} (Frequency Multiplier)

Registers \$40-\$5F, Bits 0-3\\

If MUL=0, it multiplies the operator's frequency by 0.5.  Otherwise, the
frequency is multiplied by the value in MUL (1,2,3...etc)\\

{\bfseries TL} (Total Level - attenuation)

Registers \$60-\$7F, Bits 0-6\\

This is essentially "volume control" - It is an attenuation value, so \$00 =
maximum level and \$7F is minimum level. On output operators, this is the
volume output by that operator. On modulating operators, this affects the
amount of modulation done to other operators.\\

{\bfseries KS} (Key Scaling)

Registers \$80-\$9F, Bits 6-7\\

Controls the speed of the ADSR progression. The KS value sets four different
levels of scaling. Key scaling increases along with the pitch set in KC. 0=min,
3=max\\

{\bfseries AR} (Attack Rate)

Registerss \$80-\$9F, Bits 0-4\\

Sets the attack rate of the ADSR envelope. 0=slowest, \$1F=fastest\\

{\bfseries AMS-Enable} (Amplitude Modulation Sensitivity Enable)

Registers \$A0-\$BF, Bit 7\\

If set, the operator's output level will be affected by the LFO according to
the channel's AMS setting. If clear, the operator will not be affected.\\

{\bfseries D1R} (Decay Rate 1)

Registers \$A0-\$BF, Bits 0-4\\

Controls the rate at which the level falls from peak down to the sustain level
(D1L). 0=none, \$1F=fastest.\\

{\bfseries DT2} (Detune 2 - coarse)

Registers \$C0-\$DF, Bits 6-7\\

Sets a strong detune amount to the operator's frequency. Yamaha suggests that
this is most useful for sound effects. 0=off\\

{\bfseries D2R} (Decay Rate 2)

Registers \$C0-\$DF, Bits 0-4\\

Sets the Decay2 rate, which takes effect once the level has fallen from peak
down to the sustain level (D1L). This rate continues until the level reaches
zero or until the note is released.\\

0=none, \$1F=fastest\\

{\bfseries D1L}

Registers \$E0-\$FF, Bits 4-7\\

Sets the level at which the ADSR envelope changes decay rates from D1R to D2R.
0=minimum (no D2R), \$0F=maximum (immediately at peak, which effectively
disables D1R)\\

{\bfseries RR}

Registers \$E0-\$FF, Bitst 0-3\\

Sets the rate at which the level drops to zero when a note is released. 0=none,
\$0F=fastest
