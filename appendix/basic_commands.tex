\chapter{Commander X16 BASIC}

This manual has introduced you to the BASIC language and many of the commands,
operators, and conventions.  However, that is not enough in order to truly
understand how to use BASIC.  This appendix is a reference that aims to provide
a complete documentation for Commander X16 BASIC.  It will provide the rules
(known as \emph{syntax}) of the BASIC language, and concise descriptions of
each BASIC command.\\

To make this information easier to read, it is broken up into the following
sections:\\

\begin{enumerate}

	\item {\bfseries Variables}: describes what variables are, the different
		types of variables, and the allowed variable names.

	\item {\bfseries Operators}: describes arithmetic and logical operators.

	\item {\bfseries Commands}: describes the interactive commands that are
		used to work with programs. 

	\item {\bfseries Statements}: describes the statements that can be used in
		BASIC programs.

	\item {\bfseries Functions}: describes the BASIC functions that return
		values, such as calculations and string operations.

\end{enumerate}

\section{Variables}

Variables are values that have been given names.  Programs use variables for
many purposes, and they are an important part of BASIC programming.
Programmers can \emph{assign} a value to a variable, and then use that value
later in their program by referring to the variable.  For example:\\

\codeblock{
	10 T\$ = "X16"\\
	20 PRINT T\$\\
}

The above BASIC program stores the value {\ttfamily "X16"} in a variable named
{\ttfamily T\$}, and then {\ttfamily PRINT}s the value of {\ttfamily T\$} to
the screen.\\

Variables are similar to memory addresses except for a couple of key
differences.  First, the programmer doesn't have to keep track of where a
variable is stored in the Commander X16's memory.  This job is performed by
BASIC to make the programmer's job easier.  Second, variables have a
\emph{type}.  There are three types of variables in Commander X16 BASIC.  The
three types of variables are: \emph{floating point}, \emph{integer numeric},
and \emph{string (alphanumeric)} variables.\\

\subsection{Floating Point Variables}

\emph{Floating point numeric variables} can have any value from $-10^{38}$ to
$10^{38}$, with up to nine digits of accuracy.  Floating point values can hold
partial values, such as $3.4$, $42.7$, or $0.000025$.  This makes them useful
for a variety of mathematical uses.  Floating point variables can be named with
any single letter, any letter followed by a number, or with two
letters\footnote{There are three variable names that are \emph{reserved} by
the Commander X16 for its own use, and cannot be used for variable names in
your programs.  These names are {\ttfamily ST}, {\ttfamily TI}, and {\ttfamily
TI\$}}.  For example, {\ttfamily A}, {\ttfamily A5}, or {\ttfamily AB}.\\

To assign a floating point variable, simple type your chosen name for the
variable followed by an {ttfamily =} and then the value you wish to assign it:\\

\codeblock{
	A = 3.4\\
	A5 = 42.7\\
	AB = 0.000025\\
}

For numbers that are very large or very small, you may wish to use scientific
notation to assign your variables.  The Commander X16 understands scientific
notation by using the letter {\ttfamily E} to separate the coefficient from the
exponent (the base is always assumed to be $10$).  So to assign the value $3.7
\times 10^{-14}$ to a floating point variable named {\ttfamily B2}, you would
type:\\

\codeblock{
	B2 = 3.7E-14\\
}

Not only can you assign floating point variables using scientific notation, but
the Commander X16 will also display values in scientific notation if they
require more than nine digits.\\

\subsection{Integer Variables}

\emph{Integer numeric variables} should be used whenever the number will always
be a whole number, and always be between $-32768$ and $32767$.  These are
numbers like $1$, $5$, or $-127$.  Integer variables take up less space in the
Commander X16's memory, and doing math with integers is faster than with
floating point numbers.  Integer numeric variables follow the same rules as
floating point variables, except they must have a {\ttfamily \%} character at
the end.  For example:\\

\codeblock{
	B\% = 5\\
	C5\% = -11\\
	BC\% = 1261\\
}

\note{

	Sometimes when writing numbers we place a {\ttfamily ,} to separate groups
	of three digits, such as $1,000$ or $8,006,029,545$.  While this makes
	numbers easier for humans to read, it is not something that Commander X16
	understands.  When typing numbers into your programs, you should never use
	a {\ttfamily ,} but instead type the numbers without it.  So the previous
	numbers would be typed as $1000$ and $8006029545$.\\

}

\subsection{String Variables}

\emph{String variables} are used to store characters, such as words, sentences,
or any other symbol that you can type.  A single string variable can store
either a single character, many characters in a row, or even no characters at
all!String variable names follow the same rules as floating point variables,
except they must have a {\ttfamily \$} character at the end.  The value of a
string variable must be enclosed in quotation marks.  For example:\\

\codeblock{
	N\$ = "COMMANDER X16"\\
	B8\$ = "SEVEN"\\
	DC\$ = "THE NEXT STRING HAS NO CHARACTERS IN IT"\\
	EC\$ = ""\\
}

\subsection{Arrays}

\emph{Arrays} are lists of variables that all share the same name.  You can
specify which item, or \emph{element}, in the list you are using by using a
number.  For example, if you have an array of floating point values in a
variable named {\ttfamily AB} you can use the second value in the array by
typing {\ttfamily AB(2)} where you would normally type a variable name or a
value.  You can create an array that holds any of the above types of variables,
but a single a array can only hold one type of variable.  So an array that was
created to hold seven strings can \emph{only} hold string variables, and will
cause an error if you try to assign an integer to one of the elements.\\

Unlike other variables, array variables usually\footnote{see the documentation
of the {\ttfamily DIM} statement for exceptions} need to be \emph{declared}
before using them.  You can declare your array variable with the {\ttfamily
DIM} statement like so:\\

\codeblock{
	DIM A(25)\\
}

This will tell the Commander X16 to reserve enough memory for twenty-five
floating point variables.  You can access these variables by \emph{indexing}
the array variable {\ttfamily A} when using it, like so:\\

\codeblock{
	PRINT A(14)\\
}

The above example prints the value of the fourteenth \emph{element} of
{\ttfamily A} to the screen.\\

Arrays can have more than one \emph{dimension} by declaring them with more than
one index.  For example a two-dimensional array can be useful for storing data
arranged as rows and columns.  Here is how you would declare an array with 24
rows of 32 columns:\\

\codeblock{
	DIM S\%(32,24)\\
}

The above array can store $32 \times 24$ integer values.  You could even
declare arrays with even higher dimensions if you have a need for it.  Be
warned, however, as higher dimensional arrays take up exponentially more memory
so you will quickly run out.\\

\section{Operators}

Commander X16 BASIC uses three different types of \emph{operators}:
\emph{arithmetic} operators, \emph{comparison} operators, and \emph{logical}
operators.

\subsection{Arithmetic Operators}

\emph{Arithmetic operators} are used for mathematical calculations.  Here are
the available arithmetic operators:\\

\begin{tabular}{l p{0.8\linewidth}}
	\ttfamily\bfseries{+}&addition\\
	\ttfamily\bfseries{-}&subtraction\\
	\ttfamily\bfseries{*}&multiplication\\
	\ttfamily\bfseries{/}&division\\
	\ttfamily\bfseries{↑}&raising to a power (exponentiation)\\
\end{tabular}

\vspace{16pt}

When several operators are used in the same arithmetic expression, there is an
\emph{order} in which the operations execute.  First, any exponentiation
operations execute.  Next, any multiplication or division operations execute.
Finally, any addition or subtraction operations execute.  When there are two or
more operations that execute at the same time, such as a multiplication
followed by a division, the operations execute from left to right.  Consider
the following:\\

\codeblock{
	PRINT 2/4/2\\
}

The above code will execute and print {\ttfamily .25} to the screen instead of
printing {\ttfamily 1}.  This is because {\ttfamily 2/4} executes first to
produce {\ttfamily .5}, and then {\ttfamily .5/2} executes to produce a final
value of {\ttfamily .25}.  If desired, you can force the order of operations by
enclosing calculations inside parentheses.  For example, we could reverse the
order of the operations above by typing:\\

\codeblock{
	PRINT 2/(4/2)\\
}

Now the result is {\ttfamily 1} because {\ttfamily 4/2} is executed first to
produce {\ttfamily 2}, and then {\ttfamily 2/2} executes to produce {\ttfamily
1}.

Using parentheses is a good practice even when not necessary, because it makes
the intention of the calculation obvious when reading the code.  Had we used
them in the original example, it would have made the execution obvious at first
glance:\\

\codeblock{
	PRINT (2/4)/2\\
}

The above code is identical to {\ttfamily 2/4/2}, but is easier to read.

\subsection{Comparison Operators}

\emph{Comparison operators} are useful for determining equalities and
inequalities.  These are used comparing values against each other to determine
if they are the same, not the same, or which is larger.  The comparison
operators are:\\

\begin{tabular}{l p{0.8\linewidth}}
	\ttfamily\bfseries{=}&is equal to\\
	\ttfamily\bfseries{<}&is less than\\
	\ttfamily\bfseries{>}&is greater than\\
	\ttfamily\bfseries{<= or =<}&is less than or equal to\\
	\ttfamily\bfseries{>= or =>}&is greater than or equal to\\
	\ttfamily\bfseries{<> or ><}&is not equal to\\
\end{tabular}

\vspace{16pt}

Comparison operators are most often used with {\ttfamily IF...THEN} statements.
For example:\\

\codeblock{
	A = 12\\
	IF A > 10 THEN PRINT "GREATER THAN 10"\\
}

As you can see from the code above, both variables and literal values can be
used with comparison operators.\\

\subsection{Logical Operators}

\emph{Logical operators} are used to join together multiple comparison
statements into a single statement.  There are three logical operators:\\

\begin{tabular}{l p{0.8\linewidth}}
	\ttfamily\bfseries{AND}&is true if both the left side and the right side are true\\
	\ttfamily\bfseries{OR}&is true if either the left side or the right side are true\\
	\ttfamily\bfseries{NOT}&is true if the right side is false\\
\end{tabular}

\vspace{16pt}

By using these logical operators, you can write complex conditions for your
programs.  Here's some examples:\\

\codeblock{
	IF A = B AND C = D THEN 100\\
	IF A = B OR NOT (C = D) THEN 100\\
}

Notice how parentheses can be used to explicitly force the order in which
logical conditions are evaluated, just like how they force the order in which
arithmetic is evaluated.\\

\section{Commands}

\emph{Commands} are statements in BASIC that have a use other than being part
of a BASIC program.  Commands are what you type in order to tell the Commander
X16 to do things, such as {\ttfamily LIST} the contents of the SD card,
{\ttfamily LOAD} a program from the SD card, or {\ttfamily RUN} the currently
loaded program.  Unlike other BASIC statements, commands can be used
interactively to form a sort of conversation between you and your Commander
X16.  This section contains a description of each command in alhabetical
order.\\

\subsection{CONT (continue)}

When a program has been stopped by either using the
\keybackgroundcolor{gray}\doublekey{RUN\\STOP} key, a {\ttfamily STOP}
statement, or an {\ttfamily END} statement within the program, it can be
restarted by using the {\ttfamily CONT} command.  The {\ttfamily CONT} command
will continue executing the loaded program at the exact place from which it
left off.\\

The {\ttfamily CONT} command will not always work, however.  If you make any
modifications to your program while it is stopped, the {\ttfamily CONT} command
will fail and display a {\ttfamily CAN'T CONTINUE ERROR}.  This is true even if
you {\ttfamily LIST} the program and hit \widekey{RETURN} while the cursor is
on a line of the program...even if you didn't make any modifications.  To the
X16, this is still considered a change to the program, so the only way to run
it again is to start at the beginning of the program by using the {\ttfamily
RUN} command.\\

\subsection{LIST}

The {\ttfamily LIST} command will print the currently loaded BASIC program to
the screen, either in its entirety or only the parts specified by the user.
When {\ttfamily LIST} is used without any numbers typed after it (known as
\emph{arguments}), you will see a complete listing of the program on your
screen.  If the program scrolls off the screen, and you are unable to see the
part that you want, you have a couple of options.  First, you can use the
\widekey{CTRL} key to slow down how fast lines are printed to the screen.  The
part you wish to see will still scroll off eventually, but you will be given a
much longer time to look at it.  Second, you can use the {\ttfamily LIST}
command with arguments that will limit the listing to only the line or lines
that you wish to see.  When you follow the {\ttfamily LIST} command with a
single number, the X16 will list only that line number (if it exists).  If you
follow {\ttfamily LIST} with two line numbers separated by a dash, the X16 will
list all the lines from the first number to the second (including both line
numbers).  If you follow {\ttfamily LIST} with a dash followed by a single
number, it lists from the beginning of the program up to and including the line
number.  Finally, if you follow {\ttfamily LIST} with a number followed by a
dash, it lists from the line number until the end of the program.\\

\begin{tabular}{l p{0.8\linewidth}}
	\ttfamily\bfseries{LIST}&Shows entire program.\\
	\ttfamily\bfseries{LIST 10-}&Shows only from line 10 through the end.\\
	\ttfamily\bfseries{LIST 10}&Shows only line 10.\\
	\ttfamily\bfseries{LIST -10}&Shows from the beginning through line 10.\\
	\ttfamily\bfseries{LIST 10-20}&Shows lines from 10 through 20.\\
\end{tabular}

\vspace{16pt}

\subsection{LOAD}

The {\ttfamily LOAD} command is used when you want to use a program that is
stored on the Commander X16's SD card\footnote{the {\ttfamily LOAD} command can
also be used with other devices, but only the SD card reader ships with the
Commander X16}.

\section{Statements}

\section{Functions}
