\chapter{Commander X16 BASIC}

This manual has introduced you to the BASIC language and many of the commands,
operators, and conventions.  However, that is not enough in order to truly
understand how to use BASIC.  This appendix is a reference that aims to provide
a complete documentation for Commander X16 BASIC.  It will provide the rules
(known as \emph{syntax}) of the BASIC language, and concise descriptions of
each BASIC command.\\

To make this information easier to read, it is broken up into the following
sections:\\

\begin{enumerate}

	\item {\bfseries Variables}: describes what variables are, the different
		types of variables, and the allowed variable names.

	\item {\bfseries Operators}: describes arithmetic and logical operators.

	\item {\bfseries Commands}: describes the interactive commands that are
		used to work with programs. 

	\item {\bfseries Statements}: describes the statements that can be used in
		BASIC programs.

	\item {\bfseries Functions}: describes the BASIC functions that return
		values, such as calculations and string operations.

\end{enumerate}

\section{Variables}

Variables are values that have been given names.  Programs use variables for
many purposes, and they are an important part of BASIC programming.
Programmers can \emph{assign} a value to a variable, and then use that value
later in their program by refering to the variable.  For example:\\

\codeblock{
	10 T\$ = "X16"\\
	20 PRINT T\$\\
}

The above BASIC program stores the value {\ttfamily "X16"} in a variable named
{\ttfamily T\$}, and then {\ttfamily PRINT}s the value of {\ttfamily T\$} to
the screen.\\

Variables are similar to memory addresses except for a couple of key
differences.  First, the programmer doesn't have to keep track of where a
variable is stored in the Commander X16's memory.  This job is performed by
BASIC to make the programmer's job easier.  Second, variables have a
\emph{type}.  There are three types of variables in Commander X16 BASIC.  The
three types of variables are: \emph{normal numeric}, \emph{integer numeric},
and \emph{string (alphanumeric)} variables.\\

\subsection{Normal Numeric Variables}

\emph{Normal numeric variables} are also called \emph{floating point}
variables.  They can have any value from $-10^{38}$ to $10^{38}$, with up to
nine digits of accuracy.  Normal numeric values can hold partial values, such
as $3.4$, $42.7$, or $0.000025$.  This makes them useful for a variety of
mathematical uses.  Normal numeric variables can be named with any single
letter, any letter followed by a number, or with two letters\footnote{
	There are three variable names that are \emph{reserved} by the Commander
	X16 for its own use, and cannot be used for variable names in your
	programs.  These names are {\ttfamily ST}, {\ttfamily TI}, and {\ttfamily
	TI\$}
}.  For example,
{\ttfamily A}, {\ttfamily A5}, or {\ttfamily AB}.\\

To assign a normal numeric variable, simple type your chosen name for the
variable followed by an {ttfamily =} and then the value you wish to assign it:\\

\codeblock{
	A = 3.4\\
	A5 = 42.7\\
	AB = 0.000025\\
}

For numbers that are very large or very small, you may wish to use scientic
notation to assign your variables.  The Commander X16 understands scientific
notation by using the letter {\ttfamily E} to separate the coefficient from the
exponant (the base is always assumed to be $10$).  So to assign the value $3.7
\times 10^{-14}$ to a floating point variable named {\ttfamily B2}, you would
type:\\

\codeblock{
	B2 = 3.7E-14\\
}

Not only can you assign normal numeric variables using scientific notation, but
the Commander X16 will also display values in scientific notation if they
require more than nine digits.\\

\subsection{Integer Variables}

\emph{Integer numeric variables} should be used whenever the number will always
be a whole number, and always be between $-32768$ and $32767$.  These are
numbers like $1$, $5$, or $-127$.  Integer variables take up less space in the
Commander X16's memory, and doing math with integers is faster than with
floating point numbers.  Integer numeric variables follow the same rules as
normal numeric variables, except they must have a {\ttfamily \%} character at
the end.  For example:\\

\codeblock{
	B\% = 5\\
	C5\% = -11\\
	BC\% = 1261\\
}

\note{

	Sometimes when writing numbers we place a {\ttfamily ,} to separate groups
	of three digits, such as $1,000$ or $8,006,029,545$.  While this makes
	numbers easier for humans to read, it is not something that Commander X16
	understands.  When typing numbers into your programs, you should never use
	a {\ttfamily ,} but instead type the numbers without it.  So the previous
	numbers would be typed as $1000$ and $8006029545$.\\

}

\subsection{String Variables}

\emph{String variables} are used to store characters, such as words, sentances,
or any other symbol that you can type.  A single string variable can store
either a single character, many characteers in a row, or even no characters at
all!String variable names follow the same rules as normal numeric variables,
except they must have a {\ttfamily \$} character at the end.  The value of a
string variable must be enclosed in quotation marks.  For example:\\

\codeblock{
	N\$ = "COMMANDER X16"\\
	B8\$ = "SEVEN"\\
	DC\$ = "THE NEXT STING HAS NO CHARACTERS IN IT"\\
	EC\$ = ""\\
}

\subsection{Arrays}

\emph{Arrays} are lists of variables that all share the same name.  You can
specify which item, or \emph{element}, in the list you are using by using a
number.  For example, if you have an array of normal numeric values in a
variable named {\ttfamily AB} you can use the second value in the array by
typing {\ttfamily AB(2)} where you would normally type a variable name or a
value.  You can create an array that holds any of the above types of variables,
but a single a array can only hold one type of varaible.  So an array that was
created to hold seven strings can \emph{only} hold string variables, and will
cause an error if you try to assign an integer to one of the elements.\\

Unlike other variables, array variables usually\footnote{see the documentation
of the {\ttfamily DIM} statement for exceptions} need to be \emph{declared}
before using them.  You can declare your array variable with the {\ttfamily
DIM} statement like so:\\

\codeblock{
	DIM A(25)\\
}

This will tell the Commander X16 to reserve enough memory for twenty-five
normal numeric varaibles.  You can access these variables by \emph{indexing}
the array variable {\ttfamily A} when using it, like so:\\

\codeblock{
	PRINT A(14)\\
}

The above example prints the value of the fourteenth \emph{element} of
{\ttfamily A} to the screen.\\

Arrays can have more than one \emph{dimension} by declaring them with more than
one index.  For example a two-dimensional array can be useful for storing data
arranged as rows and columns.  Here is how you would declare an array with 24
rows of 32 columns:\\

\codeblock{
	DIM S\%(32,24)\\
}

The above array can store $32 \times 24$ integer values.  You could even
declare arrays with even higher dimensions if you have a need for it.  Be
warned, however, as higher dimensional arrays take up exponentially more memory
so you will quickly run out.\\

\section{Operators}

\section{Commands}

\section{Statements}

\section{Functions}
