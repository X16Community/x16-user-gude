\chapter*{Macro Language for Music}
\addcontentsline{toc}{chapter}{\protect\numberline{}Macro Language for Music}

\section{Overview}

The play commands use a string of tokens to define sequences of notes to be
played on a single voice of the corresponding sound chip. Tokens cause various
effects to happen, such as triggering notes, changing the playback speed, etc.
In order to minimize the amount of text required to specify a sequence of
sound, the player maintains an internal state for most note parameters.

\subsection{Stateful Player Behavior}

Playback parameters such as tempo, octave, volume, note duration, etc do not
need to be specified for each note. These states are global between all voices
of both the FM and PSG sound chips. The player maintains parameter state during
and after playback. For instance, setting the octave to 5 in an {\ttfamily
FMPLAY} command will result in subsequent {\ttfamily FMPLAY} and {\ttfamily
PSGPLAY} statements beginning with the octave set to 5.\\

The player state is reset to default values whenever {\ttfamily FMINIT} or
{\ttfamily PSGINIT} are used.\\

\begin{tabular}{|c|c|c|}
	\hline
	Parameter & Default & Equivalent Token \\ \hline
	Tempo & 120 & T120 \\ \hline
	Octave & 4 & O4 \\ \hline
	Length & 4 & L4 \\ \hline
	Note Spacing & 1 & S4 \\ \hline
\end{tabular}\\

\subsection{Using Tokens}

The valid tokens are: {\ttfamily\bfseries A-G,I,K,L,O,P,R,S,T,V,<,>}.\\

Each token may be followed by optional modifiers such as numbers or symbols.
Options to a token must be given in the order they are expected, and must have
no spacing between them. Tokens may have spaces between them as desired. Any
unknown characters are ignored.\\

Example:\\

\codeblock{
	FMPLAY 0,"L4"      : REM DEFAULT LENGTH = QUARTER NOTE\\
	FMPLAY 0,"A2. C+." : REM VALID\\
	FMPLAY 0,"A.2 C.+" : REM INVALID\\
}

The valid command plays {\ttfamily\bfseries A} as a dotted half, followed by
{\ttfamily\bfseries C♭} as a dotted quarter.\\

The invalid example would play {\ttfamily\bfseries A} as a dotted quarter (not
half) because length must come before dots. Next, it would ignore the 2 as
garbage. Then it would play natural C (not sharp) as a dotted quarter. Finally,
it would ignore the + as garbage, because sharp/flat must precede length and
dot.\\

\section{Token definitions}

\subsection{Musical notes}

\begin{itemize}

	\item \bfseries{Synopsis}: Play a musical note, optionally setting the length.

	\item \bfseries{Syntax}: {\ttfamily <A-G>[<+/->][<length>][.]}

\end{itemize}

\vspace{16pt}

Example:\\

\codeblock{
	FMPLAY 0,"A+2A4C.G-8."\\
}

On the YM2151 using channel 0, plays in the current octave an
{\ttfamily\bfseries A♯} half note followed by an {\ttfamily\bfseries A} quarter
note, followed by {\ttfamily\bfseries C} dotted quarter note, followed by
{\ttfamily\bfseries G♭} dotted eighth note.\\

Lengths and dots after the note name or rest set the length just for the
current note or rest.  To set the default length for subsequent notes and
rests, use the `L` macro.\\

\subsection{Rests}

\begin{itemize}

	\item {\bfseries Synopsis}: Wait for a period of silence equal to the
		length of a note, optionally setting the length.

	\item {\bfseries Syntax}: `R[<length>][.]`

\end{itemize}

\vspace{16pt}

Example:\\

\codeblock{
	PSGPLAY 0,"CR2DRE"\\
}

On the VERA PSG using voice 0, plays in the current octave a
{\ttfamily\bfseries C} quarter note, followed by a half rest (silence),
followed by a quarter {\ttfamily\bfseries D}, followed by a quarter rest
(silence), and finally a quarter {\ttfamily\bfseries E}.

The numeral {\ttfamily 2} in {\ttfamily R2} sets the length for the {\ttfamily
R} itself but does not alter the default note length (assumed as 4 - quarter
notes in this example).

\subsection{Note Length}

\begin{itemize}

	\item {\bfseries Synopsis}: Set the default length for notes and rests that
		follow

	\item {\bfseries Syntax}: {\ttfamily L[<length>][.]}

\end{itemize}

\vspace{16pt}

Example values:\\

\begin{tabular}{l p{0.8\linewidth}}

	{\ttfamily\bfseries L4} & quarter note (crotchet)\\\\

	{\ttfamily\bfseries L16} & sixteenth note (semiquaver)\\\\

	{\ttfamily\bfseries L12} & 8th note triplets (quaver triplet)\\\\

	{\ttfamily\bfseries L4.} & dotted quarter note (1.5x the length)\\\\

	{\ttfamily\bfseries L4..} & double-dotted quarter note (1.75x the length)\\\\

\end{tabular}

\vspace{16pt}

Example program:\\

\codeblock{
	10 FMPLAY 0,"L4"\\
	20 FOR I=1 TO 2\\
	30 FMPLAY 0,"CDECL8"\\
	40 NEXT\\
}

On the YM2151 using channel 0, this program, when RUN, plays in the current
octave the sequence {\ttfamily\bfseries C} {\ttfamily\bfseries D}
{\ttfamily\bfseries E} {\ttfamily\bfseries C} first as quarter notes, then as
eighth notes the second time around.\\

\subsection{Articulation}

\begin{itemize}

	\item {\bfseries Synopsis}: Set the spacing between notes, from legato to extreme staccato

	\item {\bfseries Syntax}: {\ttfamily S<0-7>}

\end{itemize}

\vspace{16pt}

{\ttfamily S0} indicates legato. For FMPLAY, this also means that notes after
the first in a phrase don't implicitly retrigger.\\

{\ttfamily S1} is the default value, which plays a note for 7/8 of the duration
of the note, and releases the note for the remaining 1/8 of the note's
duration.\\

You can think of {\ttfamily S} is, out of 8, how much space is put between the
notes.\\

Example:\\

\codeblock{
	FMPLAY 0,"L4S1CDES0CDES4CDE"\\
}

On the YM2151 using channel 0, plays in the current octave the sequence
{\ttfamily\bfseries C} {\ttfamily\bfseries D} {\ttfamily\bfseries E} three
times, first with normal articulation, next with legato (notes all run together
and without retriggering), and finally with a moderate staccato.\\

\subsection{Explicit retrigger}

\begin{itemize}

	\item {\bfseries Synopsis}: on the YM2151, when using `S0` legato, retrigger on the next note.

	\item {\bfseries Syntax}: {\ttfamily K}

\end{itemize}

\vspace{16pt}

Example:\\

\codeblock{
	FMPLAY 0,"S0CDEKFGA"\\
}

On the YM2151 using channel 0, plays in the current octave the sequence
{\ttfamily\bfseries C} {\ttfamily\bfseries D} {\ttfamily\bfseries E} using
legato, only triggering on the first note, then the sequence
{\ttfamily\bfseries F} {\ttfamily\bfseries G} {\ttfamily\bfseries A} the same
way. The note {\ttfamily\bfseries F} is triggered without needing to release
the previous note early.\\

\subsection{Octave}

\begin{itemize}

	\item {\bfseries Synopsis}: Explictly set the octave number for notes that follow

	\item {\bfseries Syntax}: {\ttfamily O<0-7>}

\end{itemize}

\vspace{16pt}

Example:\\

\codeblock{
	PSGPLAY 0,"O4AO2AO6CDE"\\
}

On the VERA PSG using voice 0, changes to octave 4 and plays
{\ttfamily\bfseries A} (440Hz), then switches to octave 2, and plays
{\ttfamily\bfseries A} (110Hz), then switches to octave 6 and plays the
sequence {\ttfamily\bfseries C} {\ttfamily\bfseries D} {\ttfamily\bfseries
E}.\\

\subsection{Octave Up}

\begin{itemize}

	\item {\bfseries Synopsis}: Increases the octave by 1

	\item {\bfseries Syntax}: {\ttfamily >}

\end{itemize}

\vspace{16pt}

If the octave would go above 7, this macro has no effect.\\

Example:\\

\codeblock{
	PSGPLAY 0,"O4AB>C+DE"\\
}

On the VERA PSG using voice 0, changes to octave 4 and plays the first five
notes of the {\ttfamily\bfseries A major} scale by switching to octave 5
starting at the {\ttfamily\bfseries C♯}.\\

\subsection{Octave Down}

\begin{itemize}

	\item {\bfseries Synopsis}: Decreases the octave by 1

	\item {\bfseries Syntax}: {\ttfamily <}

\end{itemize}

\vspace{16pt}

If the octave would go below 0, this macro has no effect.\\

Example:\\

\codeblock{
	PSGPLAY 0,"O5GF+EDC<BAG"\\
}

On the VERA PSG using voice 0, changes to octave 5 and plays the
{\ttfamily\bfseries G major} scale from the top down by switching to octave 4
starting at the {\ttfamily\bfseries B}.\\

\subsection{Tempo}

\begin{itemize}

	\item {\bfseries Synopsis}: Sets the BPM, the number of quarter notes per minute

	\item {\bfseries Syntax}: {\ttfamily T<1-255>}

\end{itemize}

\vspace{16pt}

High tempo values and short notes tend to have inaccurate lengths due to
quantization error. Delays within a string do keep track of fractional frames
so the overall playback length should be relatively consistent.\\

Low tempo values that cause delays (lengths) to exceed 255 frames will also end
up being inaccurate. For very long notes, it may be better to use legato to
string several together.\\

Example:\\

\codeblock{
	10 FMPLAY 0,"T120C4CGGAAGR"\\
	20 FMPLAY 0,"T180C4CGGAAGR"\\
}

On the YM2151 using channel 0, plays in the current octave the first 7 notes of
{\em Twinkle Twinkle Little Star}, first at 120 beats per minute, then again
1.5 times as fast at 180 beats per minute.\\

i\subsection{Volume}

\begin{itemize}

	\item {\bfseries Synopsis}: Set the channel or voice volume

	\item {\bfseries Syntax}: {\ttfamily V<0-63>}

\end{itemize}

\vspace{16pt}

This macro mirrors the {\ttfamily PSGVOL} and {\ttfamily FMVOL} BASIC commands
for setting a channel or voice's volume. 0 is silent, 63 is maximum volume.\\

Example:\\

\codeblock{
	FMPLAY 0,"V40ECV45ECV50ECV55ECV60ECV63EC"\\
}

On the YM2151 using channel 0, starting at a moderate volume, plays the
sequence {\ttfamily\bfseries E} {\ttfamily\bfseries C}, repeatedly, increasing
the volume steadily each time.\\


\subsection{Panning}

\begin{itemize}

	\item {\bfseries Synopsis}: Sets the stereo output of a channel or voice to left, right, or both.

	\item {\bfseries Syntax}: {\ttfamily P<1-3>}

\end{itemize}

\vspace{16pt}

\begin{tabular}{l p{0.8\linewidth}}
	1 & Left\\
	2 & Right\\
	3 & Both\\
\end{tabular}

\vspace{16pt}

Example:\\

\codeblock{
	10 FOR I=1 TO 4\\
	20 PSGPLAY 0,"P1CP2B+"\\
	30 NEXT I\\
	40 PSGPLAY 0,"P3C"\\
}

On the VERA PSG using voice 0, in the current octave, repeatedly plays a
{\ttfamily\bfseries C} out of the left speaker, then a {\ttfamily\bfseries B♯}
(effectively a {\ttfamily\bfseries C} one octave higher) out of the right
speaker.  After 4 such loops, it plays a {\ttfamily\bfseries C} out of both
speakers.\\

\subsection{Instrument change}

\begin{itemize}

	\item {\bfseries Synopsis}: Sets the FM instrument (like FMINST) or PSG waveform (like PSGWAV)

	\item {\bfseries Syntax}: {\ttfamily I<0-255>} (0-162 for FM)

\end{itemize}

\vspace{16pt}

Example:\\

\codeblock{
	10 FMINIT\\
	20 FMVIB 200,15\\
	30 FMCHORD 0,"I11CI11EI11G"\\
}

This program sets up appropriate vibrato/tremolo and plays a C major chord with
the vibraphone patch across FM channels 0, 1, and 2.
