%----------------------------------------------------------------------------------------
%	PART - Getting to Know Your Commander X16
%----------------------------------------------------------------------------------------

\makeatletter\@openrightfalse
\part{Getting to Know Your Commander X16}

\outputtypein{
	\keybackgroundcolor{gray}
	\keytextcolor{black}
	1 PRINT "X16" \widekey{return}\\\\
	2 GOTO 1 \widekey{return}\\\\
	\keybackgroundcolor{black}
	\keytextcolor{white}
	\key{R}\key{U}\key{N}
	\keybackgroundcolor{gray}
	\keytextcolor{black}
	\widekey{return}
	}

\begin{tikzpicture}
	\hyphenpenalty=10000
	\bubble{2.5in}{2.5in}{5.2}{5.8}{
		This line tells the X16 to print what's between the quotation marks.}
	\bubble{2.5in}{1.75in}{4.3}{4.5}{
		This line tells the X16 to go back to Line 1 and print it again.}
	\bubble{2.5in}{1.1in}{5.0}{3.0}{
		Typing the word RUN makes the program run.}
\end{tikzpicture}

%----------------------------------------------------------------------------------------
%	CHAPTER - Getting Started
%----------------------------------------------------------------------------------------

\chapter{Getting Started}\index{Sectioning}


Congratulations!  Your Commander X16 is up and running and ready to accept your
first commands.  When it starts, it should display a message at the top letting
you know that it is running BASIC and how much memory is available.  There will
also be a white blinking rectangle called a \emph{cursor}.  This is how the X16
signals that it is waiting for you.

\section{The Start Screen}

\screenbox{2.75in}{2in}{
	**** X16 BASIC ****\\
	512k HIGH RAM\\
	38655 BASIC BYTES FREE\\\\
	READY.\\
	\cursor
}

\vspace{16pt}

\tip{
	\keybackgroundcolor{gray}
	\keytextcolor{black}

	If you type a character on the screen that you don't want, press the
	\widekey{backspace} key.  This key will erase the character immediately
	to the left of the cursor.\\

	Use this key as often as you like to delete unwanted characters.
}

\section{A Quick Experiement}

\keybackgroundcolor{white}
\keytextcolor{black}

It's time to start pressing keys and giving your Commander X16 something to do!
Press the following keys:\\

\key{p} \key{r} \key{i} \key{n} \key{t}\\



As you press each key, the cursor moves to the right.  The cursor will always
show you where the next character will be typed.  Next, locate one of the
\keybackgroundcolor{gray}\widekey{shift} keys on the keyboard.  There will be
one on the right and one on the left, but they both do the same thing: modify another
key when pressed at the same time as \widekey{shift}.\\

Hold down the \widekey{shift} key and press the
\keybackgroundcolor{white}\doublekey{"\\'} key.  The screen should now look
like this:\\

\screenbox{2.75in}{2in}{
	**** X16 BASIC ****\\
	512k HIGH RAM\\
	38655 BASIC BYTES FREE\\\\
	READY.\\
	PRINT"\cursor
}

Pressing the \doublekey{"\\'} key while holding down
\keybackgroundcolor{gray}\widekey{shift} caused the {\ttfamily "} character to
by typed instead of the {\ttfamily '} character.\\

Now let't type a word.  Without holding down any other keys, press these keys:\\

\keybackgroundcolor{white}
\key{b} \key{u} \key{t} \key{t} \key{e} \key{r} \key {f} \key{l} \key{y}\\

Finally, hold down the \keybackgroundcolor{gray}\widekey{shift} key, and press
the \keybackgroundcolor{white}\doublekey{"\\'} key one more time.  The screen
should now show:\\

{
	\raggedleft
	\screenbox{2.75in}{2in}{
		**** X16 BASIC ****\\
		512k HIGH RAM\\
		38655 BASIC BYTES FREE\\\\
		READY.\\
		PRINT"BUTTERFLY"\cursor
	}
}

\begin{tikzpicture}
	\hyphenpenalty=10000
	\smallbubble{0in}{1.1in}{4.8}{2.8}{
		Everything you typed is on this line.}
\end{tikzpicture}

If something doesn't look correct, use the
\keybackgroundcolor{gray}\widekey{backspace} key to delete characters and then
you can re-type them.\\

Once everything looks correct, find the \widekey{return} key on the keyboard
and press it once.  Now look at your screen.

\screenbox{3.5in}{2.625in}{
	**** X16 BASIC ****\\
	512k HIGH RAM\\
	38655 BASIC BYTES FREE\\\\
	READY.\\
	PRINT"BUTTERFLY"\\
	BUTTERFLY\\\\
	READY.\\
	\cursor
}

\begin{tikzpicture}
	\hyphenpenalty=10000
	\smallbubble{2.5in}{2.3in}{4.0}{4.8}{You typed this.}
	\bubble{2.5in}{1.8in}{4.5}{4.4}{
		The cursor was here when you pressed the \widekey{retrun} key.}
	% because I need 3 arrows, I just print this 3 times with different
	% settings.  The final bubble will be drawn over the others.
	\bubble{2.5in}{1.2in}{2.9}{3.9}{
		The Commander X16 printed these.}
	\bubble{2.5in}{1.2in}{2.0}{3.0}{
		The Commander X16 printed these.}
	\bubble{2.5in}{1.2in}{1.0}{2.5}{
		The Commander X16 printed these.}
\end{tikzpicture}

Pressing the \widekey{return} key told the X16 that you were finished typing
your command.  Then the X16 looked at the command you typed, saw that it was
something it knows how to do, and then did it.  In this case, your command told
the X16 to {\ttfamily PRINT} a message to the screen,  The X16 knew \emph{what}
to print because you told it that as well by placing your message between the
quotation marks.\\

When the X16 finished {\ttfamily PRINT}ing the word {\ttfamily BUTTERFLY}, it
let you know by displaying the {\ttfamily READY} message and blinking the
cursor.\\

\section{Your Own Experiments}

Now that you've {\ttfamily PRINT}ed something to the screen, try {\ttfamily
PRINT}ing other things.  Can you make the Commander X16 say {\ttfamily HELLO}?
Can you make it say your name?\\

Here are some things to keep in mind:

\begin{itemize}
	\item Make sure you spell the word {\ttfamily PRINT} correctly
	\item Put your message between quotation marks ({\ttfamily "}).  Make sure
		you have one quotation mark at the beginning and one at the end
	\item Run your command by pressing \widekey{return}
	\item If something isn't working as you expect, continuing reading to learn about errors
\end{itemize}

\section{Making A Mistake On Purpose}

What happens if you type something wrong?  Anyone who spends any amount of time
using a computer is going to mistype a command.  Let's find out what happens by
making a mistake \emph{on purpose}.  That way, we understand what is happening
when we make a mistake \emph{by accident}.  Let's make a mistake!\\

Try typing our first command, but this time misspell {\ttfamily PRINT} by
forgetting the {\ttfamily I} and typing {\ttfamily PRNT} instead:

\screenbox{2.75in}{2in}{
	READY.\\
	PRNT"BUTTERFLY"\cursor
}

This is a very easy mistake to make, and at a glace you won't even notice that
the command is wrong.  Now press \widekey{return} to run this command.  You
should see an error message:

{
	\raggedleft
	\screenbox{2.75in}{2in}{
		READY.\\
		PRNT"BUTTERFLY"\\\\
		?SYNTAX ERROR\\
		READY.\\
		\cursor
	}
}

\begin{tikzpicture}
	\hyphenpenalty=10000
	\smallbubble{0in}{1.5in}{4.9}{3.7}{The X16 lets you know that something is wrong.}
\end{tikzpicture}

Printing {\ttfamily ?SYNTAX ERROR} to the screen is how the X16 tells you that
you typed something that it does not understand.  In this case, you typed
{\ttfamily PRNT} instead of {\ttfamily PRINT}.\\

For now, don't worry about these errors.  Just do you best to you type your
commands correctly before you press \widekey{return}.

As you experiment with typing commands, the screen will scroll down to give you
more room to type and more room for the Commander X16 to print the results of
your commands.  You may want to clear the screen and bring the cursor back to
the top.  The Commander X16 has a built-in way to do this without even typing a
command:\\

Hold down the \widekey{shift} key and press the \doublekey{CLR\\HOME} key.\\

This clears the screen immediately and places the cursor at the top of the
screen.\\

\tip{
	Clearing the screen will be one of the most frequent things you do
	while working on your Commander X16.  It is worth memorizing the
	\widekey{shift} \doublekey{CLR\\HOME} key combination so that you don't
	have to reference this manual every time you want to start with a fresh
	screen.\\

	You can also clear the screen by typing the {\ttfamily CLS} command and
	pressing \widekey{return}, but most people prefer to use \widekey{shirt}
	\doublekey{CLR\\HOME}.

}

\vspace{16pt}

Now that you are comfortable typing commands, it's time to write a
\emph{series} of commands to be executed at once.  This is what is called a
\emph{computer program}.

%----------------------------------------------------------------------------------------
%	CHAPTER - Your First Computer Program
%----------------------------------------------------------------------------------------

\chapter{Your First Computer Program}\index{Sectioning}

\@openrighttrue\makeatother
