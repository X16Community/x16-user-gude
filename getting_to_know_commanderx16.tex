%----------------------------------------------------------------------------------------
%	PART - Getting to Know Your Commander X16
%----------------------------------------------------------------------------------------

\makeatletter\@openrightfalse
\part{Getting to Know Your Commander X16}

\outputtypein{
	\keybackgroundcolor{gray}
	\keytextcolor{black}
	1 PRINT "X16" \widekey{return}\\\\
	2 GOTO 1 \widekey{return}\\
}

\begin{tikzpicture}
	\hyphenpenalty=10000
	\bubble{2.5in}{2.5in}{5.2}{6.2}{
		This line tells the X16 to print what's between the quotation marks.}
	\bubble{2.5in}{1.75in}{4.3}{5.3}{
		This line tells the X16 to go back to Line 1 and print it again.}
	\bubble{2.5in}{1.1in}{6.0}{2.0}{
		Typing the word RUN makes the program run.}
\end{tikzpicture}

%----------------------------------------------------------------------------------------
%	CHAPTER - Getting Started
%----------------------------------------------------------------------------------------

\chapter{Getting Started}\index{Sectioning}


Congratulations!  Your Commander X16 is up and running and ready to accept your
first commands.  When it starts, it should display a message at the top letting
you know that it is running BASIC and how much memory is available.  There will
also be a white blinking rectangle called a \emph{cursor}.  This is how the X16
signals that it is waiting for you.

\section{The Start Screen}

\screenbox{2.75in}{2in}{
	**** X16 BASIC ****\\
	512k HIGH RAM\\
	38655 BASIC BYTES FREE\\\\
	READY.\\
	\cursor
}

\vspace{16pt}

\tip{deleting characters}{
	\keybackgroundcolor{gray}
	\keytextcolor{black}

	If you type a character on the screen that you don't want, press the
	\widekey{backspace} key.  This key will erase the character immediately
	to the left of the cursor.\\

	Use this key as often as you like to delete unwanted characters.
}

\section{A Quick Experiement}

\keybackgroundcolor{white}
\keytextcolor{black}

It's time to start pressing keys and giving your Commander X16 something to do!
Press the following keys:\\

\key{p} \key{r} \key{i} \key{n} \key{t}\\



As you press each key, the cursor moves to the right.  The cursor will always
show you where the next character will be typed.  Next, locate one of the
\keybackgroundcolor{gray}\widekey{shift} keys on the keyboard.  There will be
one on the right and one on the left, but they both do the same thing: modify another
key when pressed at the same time as \widekey{shift}.\\

Hold down the \widekey{shift} key and press the
\keybackgroundcolor{white}\doublekey{"\\'} key.  The screen should now look
like this:\\

\screenbox{2.75in}{2in}{
	**** X16 BASIC ****\\
	512k HIGH RAM\\
	38655 BASIC BYTES FREE\\\\
	READY.\\
	PRINT"\cursor
}

Pressing the \doublekey{"\\'} key while holding down
\keybackgroundcolor{gray}\widekey{shift} caused the {\ttfamily "} character to
by typed instead of the {\ttfamily '} character.\\

Now let't type a word.  Without holding down any other keys, press these keys:\\

\keybackgroundcolor{white}
\key{b} \key{u} \key{t} \key{t} \key{e} \key{r} \key {f} \key{l} \key{y}\\

Finally, hold down the \keybackgroundcolor{gray}\widekey{shift} key, and press
the \keybackgroundcolor{white}\doublekey{"\\'} key one more time.  The screen
should now show:\\

{
	\raggedleft
	\screenbox{2.75in}{2in}{
		**** X16 BASIC ****\\
		512k HIGH RAM\\
		38655 BASIC BYTES FREE\\\\
		READY.\\
		PRINT"BUTTERFLY"\cursor
	}
}

\begin{tikzpicture}
	\hyphenpenalty=10000
	\smallbubble{0in}{1.1in}{4.8}{2.8}{
		Everything you typed is on this line.}
\end{tikzpicture}

If something doesn't look correct, use the
\keybackgroundcolor{gray}\widekey{backspace} key to delete characters and then
you can re-type them.\\

Once everything looks correct, find the \widekey{return} key on the keyboard
and press it once.  Now look at your screen.

\screenbox{3.5in}{2.625in}{
	**** X16 BASIC ****\\
	512k HIGH RAM\\
	38655 BASIC BYTES FREE\\\\
	READY.\\
	PRINT"BUTTERFLY"\\
	BUTTERFLY\\\\
	READY.\\
	\cursor
}

\begin{tikzpicture}
	\hyphenpenalty=10000
	\smallbubble{2.5in}{2.3in}{4.0}{4.8}{You typed this.}
	\bubble{2.5in}{1.8in}{4.5}{4.4}{
		The cursor was here when you pressed the \widekey{return} key.}
	% because I need 3 arrows, I just print this 3 times with different
	% settings.  The final bubble will be drawn over the others.
	\bubble{2.5in}{1.2in}{2.9}{3.9}{
		The Commander X16 printed these.}
	\bubble{2.5in}{1.2in}{2.0}{3.0}{
		The Commander X16 printed these.}
	\bubble{2.5in}{1.2in}{1.0}{2.5}{
		The Commander X16 printed these.}
\end{tikzpicture}

Pressing the \widekey{return} key told the X16 that you were finished typing
your command.  Then the X16 looked at the command you typed, saw that it was
something it knows how to do, and then did it.  In this case, your command told
the X16 to {\ttfamily PRINT} a message to the screen,  The X16 knew \emph{what}
to print because you told it that as well by placing your message between the
quotation marks.\\

When the X16 finished {\ttfamily PRINT}ing the word {\ttfamily BUTTERFLY}, it
let you know by displaying the {\ttfamily READY} message and blinking the
cursor.\\

\note{

	If you are not using an official Commander X16 keyboard, then you probably
	won't have a \widekey{return} key, but instead have an \widekey{enter} key.
	Don't worry, they are the same thing.

}

\section{Your Own Experiments}

Now that you've {\ttfamily PRINT}ed something to the screen, try {\ttfamily
PRINT}ing other things.  Can you make the Commander X16 say {\ttfamily HELLO}?
Can you make it say your name?\\

Here are some things to keep in mind:

\begin{itemize}
	\item Make sure you spell the word {\ttfamily PRINT} correctly
	\item Put your message between quotation marks ({\ttfamily "}).  Make sure
		you have one quotation mark at the beginning and one at the end
	\item Run your command by pressing \widekey{return}
	\item If something isn't working as you expect, continuing reading to learn about errors
\end{itemize}

\section{Making A Mistake On Purpose}

What happens if you type something wrong?  Anyone who spends any amount of time
using a computer is going to mistype a command.  Let's find out what happens by
making a mistake \emph{on purpose}.  That way, we understand what is happening
when we make a mistake \emph{by accident}.  Let's make a mistake!\\

Try typing our first command, but this time misspell {\ttfamily PRINT} by
forgetting the {\ttfamily I} and typing {\ttfamily PRNT} instead:

\screenbox{2.75in}{2in}{
	READY.\\
	PRNT"BUTTERFLY"\cursor
}

This is a very easy mistake to make, and at a glace you won't even notice that
the command is wrong.  Now press \widekey{return} to run this command.  You
should see an error message:

{
	\raggedleft
	\screenbox{2.75in}{2in}{
		READY.\\
		PRNT"BUTTERFLY"\\\\
		?SYNTAX ERROR\\
		READY.\\
		\cursor
	}
}

\begin{tikzpicture}
	\hyphenpenalty=10000
	\smallbubble{0in}{1.5in}{4.9}{3.7}{The X16 lets you know that something is wrong.}
\end{tikzpicture}

Printing {\ttfamily ?SYNTAX ERROR} to the screen is how the X16 tells you that
you typed something that it does not understand.  In this case, you typed
{\ttfamily PRNT} instead of {\ttfamily PRINT}.\\

For now, don't worry about these errors.  Just do you best to you type your
commands correctly before you press \widekey{return}.

As you experiment with typing commands, the screen will scroll down to give you
more room to type and more room for the Commander X16 to print the results of
your commands.  You may want to clear the screen and bring the cursor back to
the top.  The Commander X16 has a built-in way to do this without even typing a
command:\\

Hold down the \widekey{shift} key and press the \doublekey{CLR\\HOME} key.\\

This clears the screen immediately and places the cursor at the top of the
screen.\\

\tip{Clearing The Screen}{
	Clearing the screen will be one of the most frequent things you do
	while working on your Commander X16.  It is worth memorizing the
	\widekey{shift} \doublekey{CLR\\HOME} key combination so that you don't
	have to reference this manual every time you want to start with a fresh
	screen.\\

	You can also clear the screen by typing the {\ttfamily CLS} command and
	pressing \widekey{return}, but most people prefer to use \widekey{shirt}
	\doublekey{CLR\\HOME}.

}

%----------------------------------------------------------------------------------------
%	CHAPTER - Your First Computer Program
%----------------------------------------------------------------------------------------

\chapter{Your First Computer Program}\index{Sectioning}

Now that you are comfortable typing commands, it's time to write a
\emph{series} of commands to be executed at once.  This is what is called a
\emph{computer program}.  Let's begin.\\

\keybackgroundcolor{gray}
\keytextcolor{black}

\begin{tabular}{l p{0.8\linewidth}}
	\bfseries STEP 1:&

	Clear the screen by holding down the \widekey{shift} key and then pressing
	the \doublekey{CLR\\HOME} key at the same time.\\

	\bfseries STEP 2:&

	Type \keybackgroundcolor{white}\key{n}\key{e}\key{w} and press the
	\keybackgroundcolor{gray}\widekey{return} key.\\

	\bfseries STEP 3:&

	Type \keybackgroundcolor{white}\key{1}\key{0} \widekey{space}
	\key{p}\key{r}\key{i}\key{n}\key{t} \widekey{space} \key{"}
	\key{x}\key{1}\key{6}\key{"}\key{;} and press
	\keybackgroundcolor{gray}\widekey{return}.\\

	\bfseries STEP 4:&

	Type \keybackgroundcolor{white}\key{2}\key{0} \widekey{space}
	\key{g}\key{o}\key{t}\key{o} \widekey{space} \key{1}\key{0} and press
	\keybackgroundcolor{gray}\widekey{return}.

\end{tabular}

\note{

	\begin{itemize}

		\item The \keybackgroundcolor{white}\widekey{space} key is the large,
			wide key at the bottom of the keyboard.  It should be the only key
			with nothing printed on it.

		\item The \key{"} key is simply the \doublekey{"\\'} key pressed while
			holding down the \keybackgroundcolor{gray}\widekey{shift} key.

		\item the \keybackgroundcolor{white}\key{;} key is the \doublekey{:\\;}
			pressed while \emph{not} holding down the
			\keybackgroundcolor{gray}\widekey{shift} key.  It is next to the
			\keybackgroundcolor{white}\doublekey{"\\'} key.

	\end{itemize}

}

When you are finished, the screen will look like this:

\begin{center}
	\screenbox{2.75in}{2in}{
		NEW\\\\
		READY.\\
		10 PRINT" X16";\\
		20 GOTO 10\\
		\cursor
	}
\end{center}

\begin{tikzpicture}
	\hyphenpenalty=10000
	\tinybubble{0in}{2.05in}{2.7}{5.2}{
		The X16 typed this.}
	\tinybubble{0in}{1.25in}{2.7}{3.7}{
		The cursor shows that the X16 is waiting.}
	\tinybubble{3.5in}{1.75in}{3.8}{6.1}{
		You typed this and then pressed \widekey{return}.}
	\tinybubble{3.5in}{1.75in}{5.8}{4.7}{
		You typed this and then pressed \widekey{return}.}
	\tinybubble{3.5in}{1.75in}{5.0}{4.2}{
		You typed this and then pressed \widekey{return}.}
\end{tikzpicture}

\tip{editing mistakes}{

	You can \emph{retype a line} anytime and the Commander X16 will replace the
	old line with the new one.  For example, if you mistyped the command
	{\ttfamily PRINT} on line 10:\\

	\codeblock{
		10 PRNNT " X16";\\
		20 GOTO 10\\
	}

	You can skip down by hitting \keybackgroundcolor{gray}\widekey{return} a
	few times and type:\\

	\codeblock{
		10 PRINT " X16";\\
	}

	Now the new line has replaced the old line in your program!  If you want to
	make sure, type \keybackgroundcolor{white}\key{l}\key{i}\key{s}\key{t} to
	tell the X16 print out your entire program to the screen.  Replacing lines
	is also a quick way for you to experiment while writing programs.\\

	Typing the line number and immediately hitting
	\keybackgroundcolor{gray}\widekey{return} will delete the entire line from
	your program.

}

If your program looks correct, it's time to tell the X16 to \emph{run} your
program.  To do this, type \keybackgroundcolor{white}\key{r}\key{u}\key{n}
\keybackgroundcolor{gray}\widekey{return}.\\

The screen should be filled with {\ttfamily X16}:\\

\begin{center}
	\screenbox{2.75in}{2in}{

		\enablehyph{X16X16X16X16X16X16X16X16X16X16X16X16X16X16X16X16X16X16X16X16X16X16X16X16X16X16X16X16X16X16X16X16X16X16X16X16X16X16X16X16X16X16X16X16X16X16X16X16X16X16X16X16X16X16X16X16X16X16X16X16X16X16X16X16X16X16X16X16X16X16X16X16X16X16X16X16X16X16X16X16X16X16X16X}

	}
\end{center}

This text is scrolling up the screen because the program is continuing to add
new text at the bottom.  The X16 allows you to slow this down by pressing the
\widekey{ctrl} key.  Just like the \widekey{shift} key, there are two of
\widekey{ctrl} keys on your keyboard; one on each side.  Holding \widekey{ctrl}
tells the X16 to reduce how fast it prints to the screen.  This is useful when
debugging programs that move too fast for your eyes to see clearly.\\

With your program running, you no longer have a cursor that is waiting for you
to type.  To stop your program and bring back the cursor, press the
\doublekey{RUN\\STOP} key.  This should stop the program, and display a
message, and then print the {\ttfamily READY} prompt followed by the cursor:

\begin{center}
	\screenbox{2.75in}{2in}{

		\enablehyph{X16X16X16X16X16X16X16X16X16X16X16X16X16X16X16X16X16X16X16X16X16X16X16X16X16X16X16X16X16X16X16X16X16X16X16X16X16X16X16X16X16X16X16X16X16X16X16X16X16X16}\\\\
		BREAK IN 10\\
		READY.\\
		\cursor

	}
\end{center}

The word {\ttfamily BREAK} is how the X16 tells you that the program has
stopped, and it also tells you which line it stopped at.  In this case, the
program \emph{broke} at line 10.  This does not mean anything is broken.  It's
just the word that the computer uses to let you know that it has stopped in the
middle of a program.\\

Now that the program has stopped running, the cursor reappears to let you know
that the X16 is waiting for you to tell it what to do.  This allows you to
change your program in some way before you run it again.  It would be nice to
be able to see your program printed to the screen so that you know what to
change.  To do this, use the {\ttfamily LIST} command by typing
\keybackgroundcolor{white}\key{l}\key{i}\key{s}\key{t}\keybackgroundcolor{gray}
\widekey{return}.  You should now see your program on the screen so that you
can make edits.  When you want to run it again, simply move the cursor to a
blank line and use the {\ttfamily RUN} command again.  Don't forget to type
\widekey{return} after you type the command!

By repeating this process of writing, running, stopping, and editing your
program, you can take your time to make your program run the way you want it to
run.  You don't have to get everything correct right away.  Even the best
computer programmers rarely get their programs to run correctly the first time
it runs.\\

\tip{editing your program}{

	When your program is {\ttfamily LIST}ed out to the screen, you can edit it
	in place by moving the cursor to the lines you want to edit.  The cursor
	can be moved by using the arrow keys on the keyboard.  Once you are on the
	line you wish to edit, you can type over top of the characters that are
	already there or use the \widekey{backspace} key to delete them and retype
	the line.  On each line you change, make sure to press the \widekey{return}
	key while on the line.  Otherwise, the Commander X16 will not replace the
	old line with the new one.

}

You have just been introduce to several aspects of the Commander X16 that you
will use in many of the later chapters.  You have:

\begin{itemize}

	\item {\ttfamily PRINT}ed messages to the screen.

	\item Cleared the screen with the \widekey{shift} and \doublekey{CLR\\HOME}
		keys.
	
	\item Written your first program and created a scrolling display.
	
	\item Slowed down the program with the \widekey{ctrl} key.

	\item Stopped the program with the \doublekey{RUN\\STOP} key.

	\item {\ttfamily LIST}ed the program.

	\item Learned ways to edit your program.

\end{itemize}

\vspace{16pt}

As you explore this guide, you will find yourself using these lessons often.
Don't worry if there are things you don't understand.  Future chapters will go
into more details about what you have learned here.  It's also important to
know that the best way to learn is by experimenting for you yourself.\\

This guide is designed so that you can go directly to \emph{any chapter} that
looks interesting to you.


\@openrighttrue\makeatother
